%
% this file is encoded in utf-8
% v2.0 (Apr. 5, 2009)
% 如果浮水印不是全篇需要,請把下列介於 >>> 與 <<<
% 的「全篇浮水印專用碼」關掉 (行首加百分號)
% 參考自 Keith Reckdahl 寫的 "Using Imported Graphics in LATEX2e" (epslatex.pdf) p.39
% 如果只有特定頁需要浮水印
% 則依該頁屬性使用下列之一的命令 
% 普通頁命令 \thispagestyle{WaterMarkPage}
% plain 頁命令 \thispagestyle{PlainWaterMarkPage}
% empty 頁命令 \thispagestyle{EmptyWaterMarkPage}


% 將重複使用的浮水印章
% 圖檔是 my_watermark.xxx
% 副檔名可以不加,可以是 latex 系統能處裡的任何格式:pdf, gif, png, jpg, eps, ...
% 某些圖檔格式在某些工作流程可能需要作前置處裡。
% 例如,pdflatex 無法直接處理 eps 檔
%  latex + dvipdfmx 無法直接處理 pdf, gif, png, jpg, 需要用 ebb 小工具程式
%  對圖檔產生 .bb 對應檔。
%
% 寬為 5.1 cm
\newsavebox{\mywatermark}
% \sbox{\mywatermark}{\includegraphics[keepaspectratio,%
% width=5.1cm]{content/figs/my_watermark}}


% 將 central header 擺放浮水印的巨集指令
\newcommand{\PlaceWaterMark}{\fancyhead[C]{\setlength{\unitlength}{1in}%
\begin{picture}(0,0)%
\put(-1,-6.2){\usebox{\mywatermark}}% 圖檔擺放的位置座標
\end{picture}}%
}

\fancyhead{}  % reset left, central, right header to empty
%% 如果不需整篇論文都要浮水印
%% 則下面  >>> 與 <<< 之間的程式碼請關閉
%% >>> 全篇浮水印
\PlaceWaterMark  % 每一頁都有浮水印 (除了 plain、empty 頁以外)

% 重新定義 plain 頁面
% 每張 plain 頁面 (每一章的第一頁) 也加浮水印

\fancypagestyle{plain}{%
\fancyhead{}%
\PlaceWaterMark%
\fancyfoot{}%
\fancyfoot[C]{\thepage}
\renewcommand{\headrulewidth}{0pt}%
\renewcommand{\footrulewidth}{0pt}%
}
%% <<< 全篇浮水印 

%% 如果只有一、兩頁需要有浮水印
%% 可以在該頁 (有頁碼) 使用 \thispagestyle{WaterMarkPage}
%% 此命令不影響原有的 header、footer
\fancypagestyle{WaterMarkPage}{%
\PlaceWaterMark%
}

%% 如果只有一、兩頁 plain 頁需要有浮水印 (如 摘要、自傳等)
%% 可以在該頁 (有頁碼) 使用 \thispagestyle{PlainWaterMarkPage}
%% 只有頁碼與浮水印,沒有其他的 header、footer
%% 等同於 plain page style + water mark
\fancypagestyle{PlainWaterMarkPage}{%
\fancyhead{}%
\PlaceWaterMark%
\fancyfoot{}%
\fancyfoot[C]{\thepage}
\renewcommand{\headrulewidth}{0pt}%
\renewcommand{\footrulewidth}{0pt}%
}

%% 如果只有一、兩頁 empty 頁需要有浮水印 (如封面、書名頁)
%% 可以在該頁 (無頁碼) 使用 \thispagestyle{EmptyWaterMarkPage}
%% 等同於 empty page style + water mark
\fancypagestyle{EmptyWaterMarkPage}{%
\fancyhead{}%
\PlaceWaterMark%
\fancyfoot{}%
\renewcommand{\headrulewidth}{0pt}%
\renewcommand{\footrulewidth}{0pt}%
}
