%
% this file is encoded in utf-8
% v2.02 (Sep. 12, 2012)
% 填入你的論文題目、姓名等資料
% 如果題目內有必須以數學模式表示的符號,請用 \mbox{} 包住數學模式,如下範例
% 如果中文名字是單名,與姓氏之間建議以全形空白填入,如下範例
% 英文名字中的稱謂,如 Prof. 以及 Dr.,其句點之後請以不斷行空白~代替一般空白,如下範例
% 如果你的指導教授沒有如預設的三位這麼多,則請把相對應的多餘教授的中文、英文名
%    的定義以空的大括號表示
%    如,\renewcommand\advisorCnameB{}
%          \renewcommand\advisorEnameB{}
%          \renewcommand\advisorCnameC{}
%          \renewcommand\advisorEnameC{}

% 論文題目 (中文)
\renewcommand\cTitle{%
Arduino 實作
}

% 論文題目 (英文)
\renewcommand\eTitle{%
Maker - Arduino 
}

% 我的姓名 (中文)
\renewcommand\myCname{開心}

% 我的姓名 (英文)
\renewcommand\myEname{Happy}

% 指導教授A的姓名 (中文)
\renewcommand\advisorCnameA{Jumbo \vspace{3cm} pro}

% 指導教授A的姓名 (英文)
\renewcommand\advisorEnameA{Jumbo, Lee}

% 指導教授B的姓名 (中文)
\renewcommand\advisorCnameB{}

% 指導教授B的姓名 (英文)
\renewcommand\advisorEnameB{}

% 指導教授C的姓名 (中文)
\renewcommand\advisorCnameC{}

% 指導教授C的姓名 (英文)
\renewcommand\advisorEnameC{}

% 校名 (中文)
\renewcommand\univCname{oo 大學}

% 校名 (英文)
\renewcommand\univEname{oo University}

% 系所名 (中文)
\renewcommand\deptCname{}

% 系所全名 (英文)
\renewcommand\fulldeptEname{}

% 系所短名 (英文, 用於書名頁學位名領域)
\renewcommand\deptEname{oo}

% 學院英文名 (如無,則以空的大括號表示)
\renewcommand\collEname{oo}

% 學位名 (中文)
\renewcommand\degreeCname{oo}

% 學位名 (英文)
\renewcommand\degreeEname{oo}

% 口試年份 (中文、民國)
\renewcommand\cYear{}

% 口試月份 (中文)
\renewcommand\cMonth{} 

% 口試年份 (阿拉伯數字、西元)
\renewcommand\eYear{} 

% 口試月份 (英文)
\renewcommand\eMonth{}

% 學校所在地 (英文)
\renewcommand\ePlace{}

%畢業級別;用於書背列印;若無此需要可忽略
\newcommand\GraduationClass{}

%%%%%%%%%%%%%%%%%%%%%%
