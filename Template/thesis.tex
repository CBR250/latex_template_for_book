%
% this file is encoded in utf-8
% v2.03 (Feb. 26, 2013)

\documentclass[12pt, a4paper]{settings/report}

% To have correct unicode mapping for special LaTeX characters (italic, ligature, ...) to unicode 
% in the generated PDF file, so that we can do copy-and-paste.
% This only works for pdflatex work flow; for latex + dvipdfmx work flow, this does not do anything
\usepackage{cmap}


% allows for temporary adjustment of side margins
\usepackage{chngpage}

% 除非校方修改了論文格式 (margins, header, footer, 浮水印, 中文數字之章別)
% 或者需要增加所用的 LaTeX 套件,
% 或者要改預設中文字型、編碼
% 否則毋須修改本檔內容
% 論文撰寫,請修改以 my_  開頭檔名的各檔案
\usepackage{CJKutf8}  %%% ZZZ %%% macro for Chinese/Japanese/Korean processing
\usepackage{CJKnumb} %%% ZZZ %%% for Chinese numbering capability
\usepackage{CJKspace} %%% ZZZ %%% suppress the spaces due to hard-coded CRs in the text, but not the spaces between CJK chars and alphabets
\usepackage[nospace]{cite}  % for smart citation
\usepackage{geometry}  % for easy margin settings
%
% margins setting
\geometry{verbose,a4paper,tmargin=3.5cm,bmargin=2cm,lmargin=4cm,rmargin=2cm}
%

% 插圖套件 graphicx
% 使用者工作流程是用 pdftex 還是 latex + dvipdfmx?
% 視情況而有不同的參數
% 這裡作自動判斷
% 參考自
% http://www.tex.ac.uk/cgi-bin/texfaq2html?label=ifpdf
\newcommand\mydvipdfmxflow{dvipdfmx}
\newcommand\mypdftexflow{pdftex}
\ifx\pdfoutput\undefined
  % not running pdftex
  \usepackage[dvipdfmx]{graphicx}
  \newcommand\myworkflow{dvipdfmx}  % set the flag for hyperref
\else
  \ifx\pdfoutput\relax
    % not running pdftex
    \usepackage[dvipdfmx]{graphicx}
    \newcommand\myworkflow{dvipdfmx}  % set the flag
  \else
    % running pdftex, with...
    \ifnum\pdfoutput>0
      % ... PDF output
      \usepackage[pdftex]{graphicx}
      \newcommand\myworkflow{pdftex}  % set the flag
    \else
      %...DVI output
      \usepackage[dvipdfmx]{graphicx}
      \newcommand\myworkflow{dvipdfmx}  % set the flag
    \fi
  \fi
\fi

% 增強功能型頁楣 / 頁腳套件
\usepackage{fancyhdr}  % 借用此套件來擺放浮水印 
% (佔用了 central header)
% 不需要浮水印的使用者仍可利用此套件,產生所需的 header, footer
%
% 啟動 fancy header/footer 套件
\pagestyle{fancy}
\fancyhead{}  % reset left, central, right header to empty
\fancyfoot[C]{\thepage} %中間 footer 擺放頁碼
\renewcommand{\headrulewidth}{0pt} % header 的直線; 0pt 則無線

% 如果不需要任何浮水印,則請把下列介於 >>> 與 <<< 之間
% 的文字行關掉 (行首加上百分號)
%% 浮水印 >>> 
%
% this file is encoded in utf-8
% v2.0 (Apr. 5, 2009)
% 如果浮水印不是全篇需要,請把下列介於 >>> 與 <<<
% 的「全篇浮水印專用碼」關掉 (行首加百分號)
% 參考自 Keith Reckdahl 寫的 "Using Imported Graphics in LATEX2e" (epslatex.pdf) p.39
% 如果只有特定頁需要浮水印
% 則依該頁屬性使用下列之一的命令 
% 普通頁命令 \thispagestyle{WaterMarkPage}
% plain 頁命令 \thispagestyle{PlainWaterMarkPage}
% empty 頁命令 \thispagestyle{EmptyWaterMarkPage}


% 將重複使用的浮水印章
% 圖檔是 my_watermark.xxx
% 副檔名可以不加,可以是 latex 系統能處裡的任何格式:pdf, gif, png, jpg, eps, ...
% 某些圖檔格式在某些工作流程可能需要作前置處裡。
% 例如,pdflatex 無法直接處理 eps 檔
%  latex + dvipdfmx 無法直接處理 pdf, gif, png, jpg, 需要用 ebb 小工具程式
%  對圖檔產生 .bb 對應檔。
%
% 寬為 5.1 cm
\newsavebox{\mywatermark}
% \sbox{\mywatermark}{\includegraphics[keepaspectratio,%
% width=5.1cm]{content/figs/my_watermark}}


% 將 central header 擺放浮水印的巨集指令
\newcommand{\PlaceWaterMark}{\fancyhead[C]{\setlength{\unitlength}{1in}%
\begin{picture}(0,0)%
\put(-1,-6.2){\usebox{\mywatermark}}% 圖檔擺放的位置座標
\end{picture}}%
}

\fancyhead{}  % reset left, central, right header to empty
%% 如果不需整篇論文都要浮水印
%% 則下面  >>> 與 <<< 之間的程式碼請關閉
%% >>> 全篇浮水印
\PlaceWaterMark  % 每一頁都有浮水印 (除了 plain、empty 頁以外)

% 重新定義 plain 頁面
% 每張 plain 頁面 (每一章的第一頁) 也加浮水印

\fancypagestyle{plain}{%
\fancyhead{}%
\PlaceWaterMark%
\fancyfoot{}%
\fancyfoot[C]{\thepage}
\renewcommand{\headrulewidth}{0pt}%
\renewcommand{\footrulewidth}{0pt}%
}
%% <<< 全篇浮水印 

%% 如果只有一、兩頁需要有浮水印
%% 可以在該頁 (有頁碼) 使用 \thispagestyle{WaterMarkPage}
%% 此命令不影響原有的 header、footer
\fancypagestyle{WaterMarkPage}{%
\PlaceWaterMark%
}

%% 如果只有一、兩頁 plain 頁需要有浮水印 (如 摘要、自傳等)
%% 可以在該頁 (有頁碼) 使用 \thispagestyle{PlainWaterMarkPage}
%% 只有頁碼與浮水印,沒有其他的 header、footer
%% 等同於 plain page style + water mark
\fancypagestyle{PlainWaterMarkPage}{%
\fancyhead{}%
\PlaceWaterMark%
\fancyfoot{}%
\fancyfoot[C]{\thepage}
\renewcommand{\headrulewidth}{0pt}%
\renewcommand{\footrulewidth}{0pt}%
}

%% 如果只有一、兩頁 empty 頁需要有浮水印 (如封面、書名頁)
%% 可以在該頁 (無頁碼) 使用 \thispagestyle{EmptyWaterMarkPage}
%% 等同於 empty page style + water mark
\fancypagestyle{EmptyWaterMarkPage}{%
\fancyhead{}%
\PlaceWaterMark%
\fancyfoot{}%
\renewcommand{\headrulewidth}{0pt}%
\renewcommand{\footrulewidth}{0pt}%
}

%% <<< 浮水印

% 如需額外的頁楣 (header) 或 footer,請在 my_headerfooter.tex 裡依例修改
% 它的預設內容是都關掉,可依需要打開
%
% this file is encoded in utf-8
% v2.0 (Apr. 5, 2009)

%%%%%%% 其他的 header (left, right) 定義
% 底下定義了一些常見的 header 型式
% 預設情況是關掉的
% 使用者可以視需要將之打開
% 也就是把下列介於 >>> 與 <<< 之間
% 的文字行打開 (行首去掉百分號)

%% header >>>
%\renewcommand{\chaptermark}[1]{%
%\markboth{\prechaptername\ \thechapter\ \postchaptername%
%\ #1}{}%
%}  %定義 header 使用的「章」層級的戳記
%\fancyhead[L]{} % 左 header 為空
%\fancyhead[R]{\leftmark}  % 右 header 擺放「章」層級的戳記 (以 \leftmark 叫出)
%\renewcommand{\headrulewidth}{0.4pt}  % header 的直線 0.4pt; 0pt 則無線
%% <<< header

%%%%%%% 其他的 footer (left, right) 定義
% 底下定義了一些常見的 footer 型式
% 預設情況是關掉的
% 使用者可以視需要將之打開
% 也就是把下列介於 >>> 與 <<< 之間
% 的文字行打開 (行首去掉百分號)

%% footer >>>
%\fancyfoot[L]{} % 左 footer 為空
%\fancyfoot[R]{\small{YZU \LaTeX\ v2.0}} % 右 footer 擺放論文格式版本
%\renewcommand{\footrulewidth}{0.4 pt} % footer 的直線 0.4pt; 0pt 則無線
%% <<< footer



%% 建立圖表的工具

\PassOptionsToPackage{usenames,dvipsnames,svgnames,table}{xcolor}
\usepackage{tikz} %TikZ and PGF are TeX packages for creating graphics programmatically
\usetikzlibrary{calc,trees,positioning,arrows,chains,shapes.geometric,%
    decorations.pathreplacing,decorations.pathmorphing,shapes,%
    matrix,shapes.symbols}



%%%%%%%%%%%%%%%%%%%%%%%%%%%%%%
%%%% 非必要的套件,但很實用
\usepackage{amsmath} % 各式 AMS 數學功能
\usepackage{amssymb} % 各式 AMS 數學符號
\usepackage{mathrsfs} %草寫體數學符號,在數學模式裡用 \mathscr{E} 得草寫 E
\usepackage{listings} % 程式列表套件
%
% listing setting
\lstset{breaklines=true,% 過長的程式行可斷行
extendedchars=false,% 中文處理不需要 extendedchars
texcl=true,% 中文註解需要有 TeX 處理過的 comment line, 所以設成 true
comment=[l]\%\%,% 以雙「百分號」做為程式中文註解的起頭標記,配合 MATLAB
basicstyle=\small,% 小號字體, 約 10 pt 大小
commentstyle=\upshape,% 預設是斜體字,會影響註解裏的英文,改用正體
%escapeinside={<>},%要在列表裡顯示中文字串,要把該中文字串用 <> 包夾住
%language=Octave % 會將一些 octave 指令以粗體顯示
}

\usepackage{url} % 在文稿中引用網址,可以用 \url{http://www.yzu.edu.tw} 方式

\usepackage{CJKfntef}  %%% ZZZ %%% CJK underline: \CJKunderline{}
\normalem %%% ZZZ %%% use \normalem to avoid \emph becomes underline, due to CJKfntef

\usepackage[version=3]{mhchem} % for easy entry of chemical formula, reaction

% the following packages are needed for table code generated by excel2latex macro in excel
\usepackage{multirow} % for sophisticated table with columns spanning multiple rows; \multirow{num_rows}{width}{contents}
\usepackage{bigstrut} % to have larger cell area
\usepackage{booktabs} % to have a more professional look of tables

% use colored text, background, foreground
% http://en.wikibooks.org/wiki/LaTeX/Colors
\usepackage{color}
\usepackage[usenames,dvipsnames,svgnames,table]{xcolor}

%%%% 以上為非必要套件
%%%%%%%%%%%%%%%%%%%%%%%%%%%%%%

%%% 以下是 hyperref 套件
%%%%%%%%%%%%%%%%%%%%%%%%%%%%%%
% hyperref 會擾亂 cite.sty 對文獻號碼縮編的排版,所以依據
% http://www.ctan.org/tex-archive/macros/latex/contrib/hyperref/
% 作如下的更動,使得 hyperref 不做文獻號碼的超連結。
\makeatletter
\def\NAT@parse{\typeout{This is a fake Natbib command to fool Hyperref.}}
\makeatother

% hyperlinkable table of contents
% 章節目錄、圖表超連結
\ifx\myworkflow\mydvipdfmxflow
	\usepackage[dvipdfmx, debug, colorlinks, linkcolor=black, citecolor=black, urlcolor=black, unicode]{hyperref}
\else
	\usepackage[pdftex, debug, colorlinks, linkcolor=black, citecolor=black, urlcolor=black, unicode]{hyperref}	
\fi

% if hyperref is not used (e.g., in LyX application)
% define dummy \phantomsection for those occurences
%   in yzu_frontpages.tex, yzu_backpages.tex, my_appendix.tex
\ifx\hypersetup\undefined
	\newcommand\phantomsection{}
\fi
%%%% 以上為所有套件
%%%% 
%%%% 


% global page layout
\newcommand{\mybaselinestretch}{1.5}  %行距 1.5 倍 + 20%, (約為 double space)
\renewcommand{\baselinestretch}{\mybaselinestretch}  % 論文行距預設值
\parskip=2ex  % 段落之間的間隔為兩個 x 的高度
\parindent = 0Pt  % 段首內縮由 CJK 控制,所以這裡就設成不內縮


% Provides support for set­ting the spac­ing be­tween lines in a document.
\usepackage{setspace}

\usepackage{framed}

% A useful extension is the subcaption[5] package which uses subfloats within a single float
\usepackage{subcaption}
% 為了讓 tabular 裡面的行與行之間可以有 分行
\usepackage{pbox}

% LaTeX默认的第一段不是首行缩进的, 这不符合我们的中文习惯.要实现首行缩进也很简单,
\usepackage{indentfirst}

%  used in a ta­ble or side­waystable en­vi­ron­ment, where \foot­note will not work 
\usepackage{tablefootnote}
%%%%%%%%%%%%%%%%%%%%%%%%%%%%%
%  end of preamble
%%%%%%%%%%%%%%%%%%%%%%%%%%%%%
%
\begin{document}
\begin{CJK*}{UTF8}{bkai}   %%% ZZZ %%%  <<< 在這裡更改預設中文字型、編碼
% CJK* 與 CJKspace 套件配合,會把原稿裏手動鍵入的換行鍵所引起的空白 (中文字與中文字之間) 拿掉
% 編碼:UTF8, Bg5, ...
% 中文字型名稱:TeXLive 安裝有一套明體字 bsmi 與 楷書 bkai; 其他字型視你的 LaTeX CJK 系統裝設情況而定

% 針對 latex + dvipdfmx 工作流程在 hyperref 套件的影響下,圖檔的辨識力退化
% 所作的權宜措施。可能是因為 TeXLive2007 hyperref 裏的
% 客製 graphicx / dvipdfmx 的設定檔不夠新
\ifx\myworkflow\mydvipdfmxflow
	\DeclareGraphicsExtensions{.pdf,.png,.jpg,.eps}
	\DeclareGraphicsRule{.pdf}{eps}{.xbb}{}
	\DeclareGraphicsRule{.png}{eps}{.xbb}{}
	\DeclareGraphicsRule{.jpg}{eps}{.xbb}{}
\fi

% global CJK setting
\CJKindent  %%% ZZZ %%%  段首內縮兩格

% 載入中文名詞的定義:例如,Figure -->「圖」, Chapter -->「第 x 章」
%
% this file is encoded in utf-8
% v2.03 (Dec. 19, 2012)

% 下列中文名詞的定義,如果以註解方式關閉取消,
% 則會以系統原先的預設值 (英文) 替代
% 名詞 \tablename 預設值為 Table
% 名詞 \figurename 預設值為 Figure
% 名詞 \lstlistingname 預設值為 Listing
\renewcommand{\tablename}{表} % 在文章中 table caption 會以「表 x」表示
\renewcommand{\figurename}{圖} % 在文章中 figure caption 會以「圖 x」表示
\renewcommand{\lstlistingname}{列表} % 在文章中 listing caption 會以「列表 x」表示


% 下列中文名詞的定義,用於論文固定的各部分之命名 (出現於目錄與該頁標題)
\newcommand{\nameInnerCover}{書名頁}
\newcommand{\nameCommitteeForm}{論文口試委員審定書}
\newcommand{\nameCopyrightForm}{授權書}
\newcommand{\nameCabstract}{中文摘要}
\newcommand{\nameEabstract}{英文摘要}
\newcommand{\nameAckn}{誌謝}
\newcommand{\nameToc}{目錄}
\newcommand{\nameLot}{表目錄}
\newcommand{\nameTof}{圖目錄}
\renewcommand{\lstlistlistingname}{程式列表目錄}
\newcommand{\nameSlist}{名詞與符號說明}
\newcommand{\nameRef}{參考文獻}
\newcommand{\nameVita}{自傳}


% 如果不需要以中文數字一、二、三呈現章別,例如「第一章」
% 則請把下列介於 >>> 與 <<< 之間
% 的文字行關掉 (行首加上百分號), 會以「第 1 章」呈現
%% 中文數字章別 >>>
%
% this file is encoded in utf-8
% v2.03 (Dec. 19, 2012)

%%% ZZZ %%% 如果不是用 LaTeX CJKnumb 套件,
%% 則必須自己製作陽春的 \CJKnumber
%% 請把下列介於 >>> 與 <<< 之間
%% 的文字行打開 (行首移除百分號)
%% 中文數字對應指令 (只能對應 1 到 10)
%% 參考自 LaTeX-CJK doc: CJK.txt, zh-Hant.cpx
%% 用法:\CJKnumber{1} 可得一

%%% 陽春 \CJKnumber 命令定義 >>>
%\makeatletter
%\@ifundefined{CJKnumber}
%  {\def\CJKnumber#1{\ifcase#1\or一\or二\or三\or四\or五\or%
% 六\or七\or八\or九\or十\fi}}{}
%\makeatother
%%% <<< 陽春 \CJKnumber 命令定義

%%%%%%%%%%%%%%%%%%%%%%%%%%%
%%% 天干對應 \CJKordinal 
\makeatletter
\@ifundefined{CJKordinal}
  {\def\CJKordinal#1{\ifcase#1\or甲\or乙\or丙\or丁\or戊\or%
 己\or庚\or辛\or壬\or癸\fi}}{}
\makeatother
%%

% 名詞 \prechaptername 預設值為 'Chapter '
% 名詞 \postchaptername 預設值為空字串

%%%%%%%%%%%%%%%%%%%%%%%%%%%%%%%%%%%%


% 請依需要選擇其中一種表現方式,把它所對應的指令列打開,其他沒有用到的表現方式的對應指令列請關閉。(用行首百分號)

%% 第一種目錄格式:
%%	1  簡介 ............................ 1
%%
%%      章別 (chapter counter) 「1」前後沒有其他文字,
%%
%%      內文章標題是
%%		第 1 章	簡介
%%	\tocprechaptername, \tocpostchaptername 都設成沒有內容的空字串
%%	\tocChNumberWidth 設成 1.4em (預設)
%%      底下三行指令請打開
%\renewcommand\tocprechaptername{}
%\renewcommand\tocpostchaptername{}
%\setlength{\tocChNumberWidth}{1.4em}


%% 第二種目錄格式:
%%	一、簡介 ............................ 1
%%
%%      章別 (chapter counter) 「一」前沒有文字,後有頓號,
%%
%%      內文章標題是
%%		第一章		簡介
%%	\tocprechaptername 設成沒有內容的空字串
%%	\tocpostchaptername 設成頓號
%%	\tocChNumberWidth 設成 2em
%%      底下六行指令請打開 (預設)
\renewcommand\countermapping[1]{\CJKnumber{#1}}
\renewcommand\tocprechaptername{}
\renewcommand\tocpostchaptername{、}
\setlength{\tocChNumberWidth}{2em}
\renewcommand\prechaptername{第} % 出現在每一章的開頭的「第x章」, x前後沒有空白
\renewcommand\postchaptername{章}


%% 第三種目錄格式:
%%	第一章、簡介 ......................... 1
%%
%%      章別 (chapter counter) 「一」前有「第」,後有「章」與頓號,
%%      內文章標題是
%%		第一章		簡介
%%	\tocprechaptername 設成「第」
%%	\tocpostchaptername 設成「章、」
%%	\tocChNumberWidth 設成 3em
%%      底下六行指令請打開
%\renewcommand\countermapping[1]{\CJKnumber{#1}}
%\renewcommand\tocprechaptername{第}
%\renewcommand\tocpostchaptername{章、}
%\setlength{\tocChNumberWidth}{3em}
%\renewcommand\prechaptername{第} % 出現在每一章的開頭的「第x章」, x前後沒有空白
%\renewcommand\postchaptername{章}


%% 第四種目錄格式:
%%	1  簡介 ............................ 1
%%
%%      章別 (chapter counter) 「1」前後沒有其他文字,
%%
%%      內文章標題是
%%		Chapter 1	簡介
%%	\tocprechaptername, \tocpostchaptername 都設成沒有內容的空字串
%%	\tocChNumberWidth 設成 1.4em (預設)
%%      底下五行指令請打開
%\renewcommand\tocprechaptername{}
%\renewcommand\tocpostchaptername{}
%\setlength{\tocChNumberWidth}{1.4em}
%\renewcommand\prechaptername{Chapter } % 出現在每一章的開頭的「Chapter x」, x前有空白
%\renewcommand\postchaptername{}


%% 可以依照需要作彈性的設定
%%
%% 在目錄裡的章別 (數字,包括後面的字串) 的寬度 \tocChNumberWidth,
%% 會影響章名與章別之間的間隔 (太少則相疊,太多則留白)
%% 建議設成 \tocpostchaptername 內容字數加一,做為 em 的倍數,
%% 但至少也要有 1.4 倍。

%% <<< 中文數字章別

%%% 以下是載入前頁、本文、後頁
% 請勿更動
% 如需針對個別章節獨立編譯
% 請在 my_chapters.tex 檔裡對個別章節的 \input 指令以行首百分號方式做開關。
%
% this file is encoded in utf-8
% v2.02 (Sep. 12, 2012)
% front matter 前頁
% 包括封面、書名頁、中文摘要、英文摘要、誌謝、目錄、表目錄、圖目錄、符號說明
% 在撰寫各章草稿時,可以把此部份「關掉」,以節省無謂的編譯時間。
% 實際內容由
%    my_names.tex, my_cabstract.tex, my_eabstract.tex, my_ackn.tex, my_symbols.tex
% 決定
% yzu_frontpages.tex 此檔只提供整體架構的定義,不需更動
% 在撰寫各章草稿時,可以把此部份「關掉」,以節省無謂的編譯時間。
%
% this file is encoded in utf-8
% v2.03 (Dec. 19, 2012)
% do not change the content of this file
% unless the thesis layout rule is changed
% 無須修改本檔內容,除非校方修改了
% 封面、書名頁、中文摘要、英文摘要、誌謝、目錄、表目錄、圖目錄、符號說明
% 等頁之格式

% make the line spacing in effect
\renewcommand{\baselinestretch}{\mybaselinestretch}
\large % it needs a font size changing command to be effective

% default variables definitions
% 此處只是預設值,不需更改此處
% 請更改 my_names.tex 內容
%   \newcommand\cTitle{論文題目}
%   \newcommand\eTitle{MY THESIS TITLE}
%   \newcommand\myCname{王鐵雄}
%   \newcommand\myEname{Aron Wang}
%   \newcommand\advisorCnameA{南宮明博士}
%   \newcommand\advisorEnameA{Dr.~Ming Nangong}
%   \newcommand\advisorCnameB{李斯坦博士}
%   \newcommand\advisorEnameB{Dr.~Stein Lee}
%   \newcommand\advisorCnameC{徐 石博士}
%   \newcommand\advisorEnameC{Dr.~Sean~Hsu}
%   \newcommand\univCname{元智大學}
%   \newcommand\univEname{Yuan Ze University}
%   \newcommand\deptCname{光電工程研究所}
%   \newcommand\fulldeptEname{Graduate School of Electro-Optical Engineering}
%   \newcommand\deptEname{Photonics Engineering}
%   \newcommand\collEname{College of Engineering}
%   \newcommand\degreeCname{碩士}
%   \newcommand\degreeEname{Master of Science}
%   \newcommand\cYear{九十四}
%   \newcommand\cMonth{六}
%   \newcommand\eYear{2006}
%   \newcommand\eMonth{June}
%   \newcommand\ePlace{Chungli, Taoyuan, Taiwan}


 % user's names; to replace those default variable definitions
% %
% this file is encoded in utf-8
% v2.02 (Sep. 12, 2012)
% 填入你的論文題目、姓名等資料
% 如果題目內有必須以數學模式表示的符號,請用 \mbox{} 包住數學模式,如下範例
% 如果中文名字是單名,與姓氏之間建議以全形空白填入,如下範例
% 英文名字中的稱謂,如 Prof. 以及 Dr.,其句點之後請以不斷行空白~代替一般空白,如下範例
% 如果你的指導教授沒有如預設的三位這麼多,則請把相對應的多餘教授的中文、英文名
%    的定義以空的大括號表示
%    如,\renewcommand\advisorCnameB{}
%          \renewcommand\advisorEnameB{}
%          \renewcommand\advisorCnameC{}
%          \renewcommand\advisorEnameC{}

% 論文題目 (中文)
\renewcommand\cTitle{%
Arduino 實作
}

% 論文題目 (英文)
\renewcommand\eTitle{%
Maker - Arduino 
}

% 我的姓名 (中文)
\renewcommand\myCname{開心}

% 我的姓名 (英文)
\renewcommand\myEname{Happy}

% 指導教授A的姓名 (中文)
\renewcommand\advisorCnameA{Jumbo \vspace{3cm} pro}

% 指導教授A的姓名 (英文)
\renewcommand\advisorEnameA{Jumbo, Lee}

% 指導教授B的姓名 (中文)
\renewcommand\advisorCnameB{}

% 指導教授B的姓名 (英文)
\renewcommand\advisorEnameB{}

% 指導教授C的姓名 (中文)
\renewcommand\advisorCnameC{}

% 指導教授C的姓名 (英文)
\renewcommand\advisorEnameC{}

% 校名 (中文)
\renewcommand\univCname{oo 大學}

% 校名 (英文)
\renewcommand\univEname{oo University}

% 系所名 (中文)
\renewcommand\deptCname{}

% 系所全名 (英文)
\renewcommand\fulldeptEname{}

% 系所短名 (英文, 用於書名頁學位名領域)
\renewcommand\deptEname{oo}

% 學院英文名 (如無,則以空的大括號表示)
\renewcommand\collEname{oo}

% 學位名 (中文)
\renewcommand\degreeCname{oo}

% 學位名 (英文)
\renewcommand\degreeEname{oo}

% 口試年份 (中文、民國)
\renewcommand\cYear{}

% 口試月份 (中文)
\renewcommand\cMonth{} 

% 口試年份 (阿拉伯數字、西元)
\renewcommand\eYear{} 

% 口試月份 (英文)
\renewcommand\eMonth{}

% 學校所在地 (英文)
\renewcommand\ePlace{}

%畢業級別;用於書背列印;若無此需要可忽略
\newcommand\GraduationClass{}

%%%%%%%%%%%%%%%%%%%%%%


%   % 使用 hyperref 在 pdf 簡介欄裡填入相關資料
%   \ifx\hypersetup\undefined
%   	\relax  % do nothing
%   \else
%   	\hypersetup{
%   	pdftitle=\cTitle,
%   	pdfauthor=\myCname}
%   \fi
%   	
%   
\newcommand\itsempty{}
%%%%%%%%%%%%%%%%%%%%%%%%%%%%%%%
%       YZU cover 封面
%%%%%%%%%%%%%%%%%%%%%%%%%%%%%%%
%
\begin{titlepage}
% no page number
% next page will be page 1

% aligned to the center of the page
\begin{center}
%          % font size (relative to 12 pt):
%          % \large (14pt) < \Large (18pt) < \LARGE (20pt) < \huge (24pt)< \Huge (24 pt)
%          %
%          %           \makebox[6cm][s]{\Huge\univCname}\\  %顯示中文校名
%          %           \vspace{1.5cm}
%          %           \makebox[12cm][s]{\Huge\deptCname}\\ %顯示中文系所名
%          %           \vspace{1.5cm}
%          %           \makebox[6cm][s]{\Huge\degreeCname 論文}\\ %顯示論文種類 (中文)
%          %           \vspace{1.5cm}
%           %
%           % Set the line spacing to single for the titles (to compress the lines)
%           \renewcommand{\baselinestretch}{1}   %行距 1 倍
%           %\large % it needs a font size changing command to be effective
%          %           \Large\cTitle\\  % 中文題目
%          %           %
%          %           \vspace{1cm}
%          %           %
%          %           \Large\eTitle\\ %英文題目
%           \vspace{2cm}
%           % \makebox is a text box with specified width;
%           % option s: stretch; option l: left aligned
%           % use \makebox to make sure
%           % 「研究生」 與「指導教授」occupy the same width
%           % Names are filled in a box with pre-defined width
%           % the left and right sides of 「:」occupy the same width (use \hspace{} to fill the short)
%           % to guarantee 「:」is at the center
%           % assume the width of a Chinese character is 1.2em
%           % 4.8em is determined by the length of the longest string "指導教授"
%           % 7.2em is determined by the length of the possibly longest name + title "歐陽明志博士"
%           \hspace{2.4em}%
%           \makebox[4.8em][s]{\Large 研究生}%
%           \makebox[1em][c]{\Large :}%
%          %           \makebox[7.2em][l]{\Large\myCname}\\  % 顯示作者中文名
%           %
%           \hspace{2.4em}%
%           \makebox[4.8em][s]{\Large 指導教授}%
%           \makebox[1em][c]{\Large :}%
%          %           \makebox[7.2em][l]{\Large\advisorCnameA}\\  %顯示指導教授A中文名
%           %
%           % 判斷是否有共同指導的教授 B
%           \ifx \advisorCnameB  \itsempty
%           \relax % 沒有 B 教授,所以不佔版面,不印任何空白
%           \else
%           % 共同指導的教授 B
%           \hspace{2.4em}%
%           \makebox[4.8em][s]{}%
%           \makebox[1em][c]{}%
%           %          \makebox[7.2em][l]{\Large\advisorCnameB}\\%顯示指導教授B中文名
%           \fi
%           %
%           % 判斷是否有共同指導的教授 C
%           \ifx \advisorCnameC  \itsempty
%           \relax % 沒有 C 教授,所以不佔版面,不印任何空白
%           \else
%           % 共同指導的教授 C
%           \hspace{2.4em}%
%           \makebox[4.8em][s]{}%
%           \makebox[1em][c]{}%
%          %           \makebox[7.2em][l]{\Large\advisorCnameC}\\%顯示指導教授B中文名
%           \fi
%           %
%           \vfill
%           %          \makebox[10cm][s]{\Large 中華民國\cYear 年\cMonth 月}%
%          
\end{center}
% Resume the line spacing to the desired setting
\renewcommand{\baselinestretch}{\mybaselinestretch}   %恢復原設定
% it needs a font size changing command to be effective
% restore the font size to normal
\normalsize
\end{titlepage}
 %%%%%%%%%%%%%%
%   
%   %% 從摘要到本文之前的部份以小寫羅馬數字印頁碼
%   % 但是從「書名頁」(但不印頁碼) 就開始計算
%   \setcounter{page}{1}
%   \pagenumbering{roman}
%   %%%%%%%%%%%%%%%%%%%%%%%%%%%%%%%
%   %       書名頁 
%   %%%%%%%%%%%%%%%%%%%%%%%%%%%%%%%
%   %
%   \newpage
%   
%   % 判斷是否要浮水印?
%   \ifx\mywatermark\undefined 
%     \thispagestyle{empty}  % 無頁碼、無 header (無浮水印)
%   \else
%     \thispagestyle{EmptyWaterMarkPage} % 無頁碼、有浮水印
%   \fi
%   
%   %no page number
%   % create an entry in table of contents for 書名頁
%   \phantomsection % for hyperref to register this
%   \addcontentsline{toc}{chapter}{\nameInnerCover}
%   
%   
%   % aligned to the center of the page
%   \begin{center}
%   % font size (relative to 12 pt):
%   % \large (14pt) < \Large (18pt) < \LARGE (20pt) < \huge (24pt)< \Huge (24 pt)
%   % Set the line spacing to single for the titles (to compress the lines)
%   \renewcommand{\baselinestretch}{1}   %行距 1 倍
%   % it needs a font size changing command to be effective
%   %中文題目
%   \Large\cTitle\\ %%%%%
%   \vspace{1cm}
%   % 英文題目
%   \Large\eTitle\\ %%%%%
%   %\vspace{1cm}
%   \vfill
%   % \makebox is a text box with specified width;
%   % option s: stretch
%   % use \makebox to make sure
%   % 「研究生:」 與「指導教授:」occupy the same width
%   \large %to have correct em value
%   \makebox[4.8em][s]{研究生}%
%   \makebox[1em][c]{:}%
%   \makebox[7.2em][l]{\myCname}%%%%%
%   \hfill%
%   \makebox[2cm][l]{Student:}%
%   \makebox[5cm][l]{\myEname}\\ %%%%%
%   %
%   %\vspace{1cm}
%   %
%   \makebox[4.8em][s]{指導教授}%
%   \makebox[1em][c]{:}%
%   \makebox[7.2em][l]{\advisorCnameA}%%%%%
%   \hfill%
%   \makebox[2cm][l]{Advisor:}%
%   \makebox[5cm][l]{\advisorEnameA}\\ %%%%%
%   %
%   % 判斷是否有共同指導的教授 B
%   \ifx \advisorCnameB  \itsempty
%   \relax % 沒有 B 教授,所以不佔版面,不印任何空白
%   \else
%   %共同指導的教授B
%   \makebox[4.8em][s]{}%
%   \makebox[1em][c]{}%
%   \makebox[7.2em][l]{\advisorCnameB}%%%%%
%   \hfill%
%   \makebox[2cm][l]{}%
%   \makebox[5cm][l]{\advisorEnameB}\\ %%%%%
%   \fi
%   %
%   % 判斷是否有共同指導的教授 C
%   \ifx \advisorCnameC  \itsempty
%   \relax % 沒有 C 教授,所以不佔版面,不印任何空白
%   \else
%   %共同指導的教授C
%   \makebox[4.8em][s]{}%
%   \makebox[1em][c]{}%
%   \makebox[7.2em][l]{\advisorCnameC}%%%%%
%   \hfill%
%   \makebox[2cm][s]{}%
%   \makebox[5cm][l]{\advisorEnameC}\\ %%%%%
%   \fi
%   %
%   % Resume the line spacing to the desired setting
%   \renewcommand{\baselinestretch}{\mybaselinestretch}   %恢復原設定
%   \normalsize %it needs a font size changing command to be effective
%   \large
%   %
%   \vfill
%   \makebox[4cm][s]{\univCname}\\% 校名
%   \makebox[6cm][s]{\deptCname}\\% 系所名
%   \makebox[3cm][s]{\degreeCname 論文}\\% 學位名
%   %
%   %\vspace{1cm}
%   \vfill
%   \large
%   A Thesis\\%
%   Submitted to %
%   %
%   \fulldeptEname\\%系所全名 (英文)
%   %
%   %
%   \ifx \collEname  \itsempty
%   \relax % 沒有學院名 (英文),所以不佔版面,不印任何空白
%   \else
%   % 有學院名 (英文)
%   \collEname\\% 學院名 (英文)
%   \fi
%   %
%   \univEname\\%校名 (英文)
%   %
%   in Partial Fulfillment of the Requirements\\
%   %
%   for the Degree of\\
%   %
%   \degreeEname\\%學位名(英文)
%   %
%   in\\
%   %
%   \deptEname\\%系所短名(英文;表明學位領域)
%   %
%   \eMonth\ \eYear\\%月、年 (英文)
%   %
%   \ePlace% 學校所在地 (英文)
%   \vfill
%   中華民國%
%   \cYear% %%%%%
%   年%
%   \cMonth% %%%%%
%   月\\
%   \end{center}
%   % restore the font size to normal
%   \normalsize
%   \clearpage
%   %%%%%%%%%%%%%%%%%%%%%%%%%%%%%%%
%   %       論文口試委員審定書 (計頁碼,但不印頁碼) 
%   %%%%%%%%%%%%%%%%%%%%%%%%%%%%%%%
%   %
%   % insert the printed standard form when the thesis is ready to bind
%   % 在口試完成後,再將已簽名的審定書放入以便裝訂
%   % create an entry in table of contents for 審定書
%   % 目前送出空白頁
%   \newpage%
%   {\thispagestyle{empty}%
%   \phantomsection % for hyperref to register this
%   \addcontentsline{toc}{chapter}{\nameCommitteeForm}%
%   %\mbox{}\clearpage
%   }
%   
%   %%%%%%%%%%%%%%%%%%%%%%%%%%%%%%%
%   %       授權書 (計頁碼,但不印頁碼) 
%   %%%%%%%%%%%%%%%%%%%%%%%%%%%%%%%
%   %
%   % insert the printed standard form when the thesis is ready to bind
%   % 在口試完成後,再將已簽名的授權書放入以便裝訂
%   % create an entry in table of contents for 授權書
%   % 目前送出空白頁
%   % (Oct. 31, 2012 新規定,授權書有四頁,這裏會送出四張空白頁)
%   \newpage% 
%   {\thispagestyle{empty}%
%   \phantomsection % for hyperref to register this
%   \addcontentsline{toc}{chapter}{\nameCopyrightForm}%
%   %\mbox{}\clearpage
%   }
%   
%   \newpage% 2nd 
%   {\thispagestyle{empty}%
%   %\mbox{}\clearpage
%   }
%   
%   \newpage% 3rd
%   {\thispagestyle{empty}%
%   %\mbox{}\clearpage
%   }
%   
%   \newpage% 4th
%   {\thispagestyle{empty}%
%   %\mbox{}\clearpage
%   }
%   
%   %%%%%%%%%%%%%%%%%%%%%%%%%%%%%%%
%   %       中文摘要 
%   %%%%%%%%%%%%%%%%%%%%%%%%%%%%%%%
%   %
%   \newpage
%   \thispagestyle{plain}  % 無 header,但在浮水印模式下會有浮水印
%   % create an entry in table of contents for 中文摘要
%   \phantomsection % for hyperref to register this
%   \addcontentsline{toc}{chapter}{\nameCabstract}
%   
%   % aligned to the center of the page
%   \begin{center}
%   % font size (relative to 12 pt):
%   % \large (14pt) < \Large (18pt) < \LARGE (20pt) < \huge (24pt)< \Huge (24 pt)
%   % Set the line spacing to single for the names (to compress the lines)
%   \renewcommand{\baselinestretch}{1}   %行距 1 倍
%   % it needs a font size changing command to be effective
%   \large\cTitle\\  %中文題目
%   \vspace{0.83cm}
%   % \makebox is a text box with specified width;
%   % option s: stretch
%   % use \makebox to make sure
%   % each text field occupies the same width
%   \hspace{50pt}
%   \makebox[3em][l]{學生:}%
%   \makebox[4.8em][l]{\myCname}%學生中文姓名
%   \hfill%
%   %
%   \makebox[5em][l]{指導教授:}%
%   \makebox[7.2em][l]{\advisorCnameA}\\ %教授A中文姓名
%   %
%   % 判斷是否有共同指導的教授 B
%   \ifx \advisorCnameB  \itsempty
%   \relax % 沒有 B 教授,所以不佔版面,不印任何空白
%   \else
%   %共同指導的教授B
%   \makebox[3em][l]{}%
%   \makebox[4.8em][l]{}%%%%%
%   \hfill%
%   \makebox[5em][l]{}%
%   \makebox[7.2em][l]{\advisorCnameB}\\ %教授B中文姓名
%   \fi
%   %
%   % 判斷是否有共同指導的教授 C
%   \ifx \advisorCnameC  \itsempty
%   \relax % 沒有 C 教授,所以不佔版面,不印任何空白
%   \else
%   %共同指導的教授C
%   \makebox[3em][l]{}%
%   \makebox[4.8em][l]{}%%%%%
%   \hfill%
%   \makebox[5em][l]{}%
%   \makebox[7.2em][l]{\advisorCnameC}\\ %教授B中文姓名
%   \fi
%   %
%   \vspace{0.42cm}
%   %
%   \univCname\deptCname\\ %校名系所名
%   \vspace{0.83cm}
%   %\vfill
%   \makebox[2.5cm][s]{摘要}\\
%   \end{center}
%   % Resume the line spacing to the desired setting
%   \renewcommand{\baselinestretch}{\mybaselinestretch}   %恢復原設定
%   %it needs a font size changing command to be effective
%   % restore the font size to normal
%   \normalsize
%   %%%%%%%%%%%%%
%   

%   
%   %%%%%%%%%%%%%%%%%%%%%%%%%%%%%%%
%   %       英文摘要 
%   %%%%%%%%%%%%%%%%%%%%%%%%%%%%%%%
%   %
%   \newpage
%   \thispagestyle{plain}  % 無 header,但在浮水印模式下會有浮水印
%   
%   % create an entry in table of contents for 英文摘要
%   \phantomsection % for hyperref to register this
%   \addcontentsline{toc}{chapter}{\nameEabstract}
%   
%   % aligned to the center of the page
%   \begin{center}
%   % font size:
%   % \large (14pt) < \Large (18pt) < \LARGE (20pt) < \huge (24pt)< \Huge (24 pt)
%   % Set the line spacing to single for the names (to compress the lines)
%   \renewcommand{\baselinestretch}{1}   %行距 1 倍
%   %\large % it needs a font size changing command to be effective
%   \hspace{-50pt}\large\eTitle\\  %英文題目
%   \vspace{0.83cm}
%   % \makebox is a text box with specified width;
%   % option s: stretch
%   % use \makebox to make sure
%   % each text field occupies the same width
%   \makebox[2cm][l]{Student:}%
%   \makebox[5cm][l]{\myEname}%學生英文姓名
%   \hfill%
%   %
%   \makebox[2cm][l]{Advisor:}%
%   \makebox[5cm][l]{\advisorEnameA}\\ %教授A英文姓名
%   %
%   % 判斷是否有共同指導的教授 B
%   \ifx \advisorCnameB  \itsempty
%   \relax % 沒有 B 教授,所以不佔版面,不印任何空白
%   \else
%   %共同指導的教授B
%   \makebox[2cm][l]{}%
%   \makebox[5cm][l]{}%%%%%
%   \hfill%
%   \makebox[2cm][l]{}%
%   \makebox[5cm][l]{\advisorEnameB}\\ %教授B英文姓名
%   \fi
%   %
%   % 判斷是否有共同指導的教授 C
%   \ifx \advisorCnameC  \itsempty
%   \relax % 沒有 C 教授,所以不佔版面,不印任何空白
%   \else
%   %共同指導的教授C
%   \makebox[2cm][l]{}%
%   \makebox[5cm][l]{}%%%%%
%   \hfill%
%   \makebox[2cm][l]{}%
%   \makebox[5cm][l]{\advisorEnameC}\\ %教授C英文姓名
%   \fi
%   %
%   \vspace{0.42cm}
%   Submitted to \fulldeptEname\\  %英文系所全名
%   %
%   \ifx \collEname  \itsempty
%   \relax % 如果沒有學院名 (英文),則不佔版面,不印任何空白
%   \else
%   % 有學院名 (英文)
%   \collEname\\% 學院名 (英文)
%   \fi
%   %
%   \univEname\\  %英文校名
%   \vspace{0.83cm}
%   %\vfill
%   %
%   ABSTRACT\\
%   %\vspace{0.5cm}
%   \end{center}
%   % Resume the line spacing the desired setting
%   \renewcommand{\baselinestretch}{\mybaselinestretch}   %恢復原設定
%   %\large %it needs a font size changing command to be effective
%   % restore the font size to normal
%   \normalsize
%   %%%%%%%%%%%%%
%   The abstract written in English is placed here.


%   %%%%%%%%%%%%%%%%%%%%%%%%%%%%%%%
%   %       誌謝 
%   %%%%%%%%%%%%%%%%%%%%%%%%%%%%%%%
%   %
%   % Acknowledgment
%   \newpage
%   \chapter*{\protect\makebox[5cm][s]{\nameAckn}} %\makebox{} is fragile; need protect
%   \phantomsection % for hyperref to register this
%   \addcontentsline{toc}{chapter}{\nameAckn}
%   

\begin{flushright}
\text{oo 謹誌} \\
\text{西元四零零零年 六月} \\
\text{于oo大學 Happy Class} 
\end{flushright}


%   
%%%%%%%%%%%%%%%%%%%%%%%%%%%%%%%
%       目錄 
%%%%%%%%%%%%%%%%%%%%%%%%%%%%%%%
%
% Table of contents
\newpage
\renewcommand{\contentsname}{\protect\makebox[5cm][s]{\nameToc}}
%\makebox{} is fragile; need protect
\phantomsection % for hyperref to register this
\addcontentsline{toc}{chapter}{\nameToc}

%\tableofcontents
\begin{spacing}{1}
   \tableofcontents
\end{spacing}
%%%%%%%%%%%%%%%%%%%%%%%%%%%%%%%
%       表目錄 
%%%%%%%%%%%%%%%%%%%%%%%%%%%%%%%
%
% List of Tables
\newpage
\renewcommand{\listtablename}{\protect\makebox[5cm][s]{\nameLot}}
%\makebox{} is fragile; need protect
\phantomsection % for hyperref to register this
\addcontentsline{toc}{chapter}{\nameLot}
%\listoftables

\begin{spacing}{1}
   \listoftables
\end{spacing}
%%%%%%%%%%%%%%%%%%%%%%%%%%%%%%%
%       圖目錄 
%%%%%%%%%%%%%%%%%%%%%%%%%%%%%%%
%
% List of Figures
\newpage
\renewcommand{\listfigurename}{\protect\makebox[5cm][s]{\nameTof}}
%\makebox{} is fragile; need protect
\phantomsection % for hyperref to register this
\addcontentsline{toc}{chapter}{\nameTof}
%\listoffigures

\begin{spacing}{1}
   \listoffigures
\end{spacing}
%%%%%%%%%%%%%%%%%%%%%%%%%%%%%%%
%       程式列表目錄 
%%%%%%%%%%%%%%%%%%%%%%%%%%%%%%%
%
% List of Listings
% 如果需要程式列表目錄,則以下四行每一行的行首的註釋符號 % 刪掉以解除封印
%\newpage
%\phantomsection % for hyperref to register this
%\addcontentsline{toc}{chapter}{\lstlistlistingname}
%\lstlistoflistings

%%%%%%%%%%%%%%%%%%%%%%%%%%%%%%%
%       符號說明 
%%%%%%%%%%%%%%%%%%%%%%%%%%%%%%%
%
% Symbol list
% define new environment, based on standard description environment
% adapted from p.60~64, <<The LaTeX Companion>>, 1994, ISBN 0-201-54199-8
\newcommand{\SymEntryLabel}[1]%
  {\makebox[3cm][l]{#1}}
%
\newenvironment{SymEntry}
   {\begin{list}{}%
       {\renewcommand{\makelabel}{\SymEntryLabel}%
        \setlength{\labelwidth}{3cm}%
        \setlength{\leftmargin}{\labelwidth}%
        }%
   }%
   {\end{list}}
%%
%%\newpage
%%\chapter*{\protect\makebox[5cm][s]{\nameSlist}} %\makebox{} is fragile; need protect
%%\phantomsection % for hyperref to register this
%%\addcontentsline{toc}{chapter}{\nameSlist}
%%%
% this file is encoded in utf-8
% v2.0 (Apr. 5, 2009)
%  各符號以 \item[] 包住,然後接著寫說明
% 如果符號是數學符號,應以數學模式$$表示,以取得正確的字體
% 如果符號本身帶有方括號,則此符號可以用大括號 {} 包住保護
\begin{SymEntry}

\item[OLED]
Organic Light Emitting Diode

\item[$E$]
energy

\item[$e$]
the absolute value of the electron charge, $1.60\times10^{-19}\,\text{C}$
 
\item[$\mathscr{E}$]
electric field strength (V/cm)

\item[{$A[i,j]$}]
the  element of the matrix $A$ at $i$-th row, $j$-th column\\
矩陣 $A$ 的第 $i$ 列,第 $j$ 行的元素

\end{SymEntry}


\newpage
%% 論文本體頁碼回復為阿拉伯數字計頁,並從頭起算
\pagenumbering{arabic}
%%%%%%%%%%%%%%%%%%%%%%%%%%%%%%%%
 

% 載入自訂巨集定義檔
% 這裡定義的巨集的使用說明,可以參考 example_macro_demo.pdf
% 預設載入。
% 如果不想要使用,把下一行行首加入百分號。
%% non-float version of table and figure environments
\makeatletter
\newenvironment{tablehere}
  {\def\@captype{table}}
  {}

\newenvironment{figurehere}
  {\def\@captype{figure}}
  {}
\makeatother


% some useful macros
%

% a single figure
%
% usage:
% \fig{width}
% {path/filename}
% {caption}
% {label}
\newcommand{\fig}[4]{
\begin{figure}[tbph]
{\centering \includegraphics[%
  width=#1,%
  keepaspectratio]{#2}
 \caption{#3}
 \label{#4}
 \par
}
\end{figure}
}

% a single figure, with additional text beneath the caption.  Usually the source of the figure.
%
% usage:
% \figt{width}
% {path/filename}
% {caption}
% {label}
% {additional text beneath the caption.  Usually the source of the figure.}
\newcommand{\figt}[5]{
\begin{figure}[tbph]
{\centering \includegraphics[%
  width=#1,%
  keepaspectratio]{#2}
 \caption{#3}
 \label{#4}
 #5\par
 }
\end{figure}
}

% single figure, additional text below the caption, spacing control between the figure and the caption
%
% usage:
% \figts{width}
% {path/filename}
% {caption}
% {label}
% {additional text beneath the caption.  Usually the source of the figure.}
% {place negative height, say, -4ex to reduce the gap between fig and caption}
\newcommand{\figts}[6]{
\begin{figure}[tbph]
{\centering \includegraphics[%
  width=#1,%
  keepaspectratio]{#2}
  \vspace{#6}\par
 \caption{#3}
 \label{#4}
 #5\par
 }
\end{figure}
}

% single figure, additional text below the caption, spacing control between the figure and the caption, trimming
%
% usage:
% \figtsr{width}
% {path/filename}
% {caption}
% {label}
% {additional text beneath the caption.  Usually the source of the figure.}
% {place negative height, say, -4ex to reduce the gap between fig and caption}
% {left bottom right top} in unit of bp (=1/72 in)
\newcommand{\figtsr}[7]{
\begin{figure}[tbph]
{\centering \includegraphics[%
  width=#1,%
  keepaspectratio,%
  trim=#7,clip=true]{#2}
  \vspace{#6}\par
 \caption{#3}
 \label{#4}
 #5\par
 }
\end{figure}
}

% multiple figures and captions in side-by-side arrangement;
%
% usage: (must be within figure environment)
% \mpfig{width}% in terms of \columnwidth
% {path/filename}
% {caption}
% {label}
\newcommand{\mpfig}[4]{
\begin{minipage}[b][1\totalheight]{#1\columnwidth}
\par\vspace{0pt}
{\centering \includegraphics[width=1\columnwidth,keepaspectratio]{#2}
\caption{#3}
\label{#4}\par}
\end{minipage}
}

% multiple figures and captions in side-by-side arrangement;
% each has its own descriptive text beneath the caption.
%
% usage: (must be within figure environment)
% \mpfigt{width}% in terms of \columnwidth
% {path/filename}
% {caption}
% {label}
% {additional text beneath the caption.  Usually the source of the figure.}
\newcommand{\mpfigt}[5]{
\begin{minipage}[b][1\totalheight]{#1\columnwidth}
{\centering \includegraphics[width=1\columnwidth,keepaspectratio]{#2}
\caption{#3}
\label{#4}
#5\par}
\end{minipage}
}

% multiple figures and captions in side-by-side arrangement, with spacing gap control;
%
% usage: (must be within figure environment)
% \mpfigts{width}% in terms of \columnwidth
% {path/filename}
% {caption}
% {label}
% {additional text beneath the caption.  Usually the source of the figure.}
% {place negative height, say, -4ex to reduce the gap between fig and caption}
\newcommand{\mpfigts}[6]{
\begin{minipage}[b][1\totalheight]{#1\columnwidth}
\par\vspace{0pt}
{\centering \includegraphics[width=1\columnwidth,keepaspectratio]{#2}
\vspace{#6}\par
\caption{#3}
\label{#4}
#5\par}
\end{minipage}
}

% multiple figures and captions in side-by-side arrangement, with spacing, w/ trim;
%
% usage: (must be within figure environment)
% \mpfigtra{width}% in terms of \columnwidth
% {path/filename}
% {caption}
% {label}
% {additional text beneath the caption.  Usually the source of the figure.}
% {place negative height, say, -4ex to reduce the gap between fig and caption}
% {left bottom right top} in terms of bp (=1/72 in)
\newcommand{\mpfigtsr}[7]{
\begin{minipage}[b][1\totalheight]{#1\columnwidth}
\par\vspace{0pt}
{\centering \includegraphics[width=1\columnwidth,keepaspectratio,%
 trim=#7,clip=true]{#2}
\vspace{#6}\par
\caption{#3}
\label{#4}
#5\par}
\end{minipage}
}

% multiple figures and captions in side-by-side arrangement, w/ spacing, w/ trim, w/ alignment;
%
% usage: (must be within figure environment)
% \mpfigtsra{width}% in terms of \columnwidth
% {path/filename}
% {caption}
% {label}
% {additional text beneath the caption.  Usually the source of the figure.}
% {place negative height, say, -4ex to reduce the gap between fig and caption}
% {left bottom right top}
% {alignment}
\newcommand{\mpfigtsra}[8]{
\begin{minipage}[#8][1\totalheight]{#1\columnwidth}
\par\vspace{0pt}
{\centering \includegraphics[width=1\columnwidth,keepaspectratio,%
 trim=#7,clip=true]{#2}
\vspace{#6}\par
\caption{#3}
\label{#4}
#5\par}
\end{minipage}
}

% multiple figures in side-by-side arrangement, but with a single caption
%
% usage: (must be within figure environment)
%     the caption capability is not included in this macro
%
% \mpfigabc{width}% in terms of \columnwidth
% {path/filename}
% {text beneath the figure.  Usually for the numbering of the subfigure}
\newcommand{\mpfigabc}[3]{
\begin{minipage}[b][1\totalheight]{#1\columnwidth}
\par\vspace{0pt}
{\centering \includegraphics[width=1\columnwidth,keepaspectratio]{#2}
#3\par}
\end{minipage}
}

% multiple figures in side-by-side arrangement, but with a single caption, w/spacing
%
% usage: (must be within figure environment)
%     the caption capability is not included in this macro
%
% \mpfigabcs{width}% in terms of \columnwidth
% {path/filename}
% {text beneath the figure.  Usually for the numbering of the subfigure}
% {place negative height, say, -4ex to reduce the gap between fig and caption}
\newcommand{\mpfigabcs}[4]{
\begin{minipage}[b][1\totalheight]{#1\columnwidth}
\par\vspace{0pt}
{\centering \includegraphics[width=1\columnwidth,keepaspectratio]{#2}\par
\vspace*{#4}#3\par}
\end{minipage}
}

% multiple figures in side-by-side arrangement, but with a single caption, w/spacing, w/trimming
%
% usage: (must be within figure environment)
%     the caption capability is not included in this macro
%
% \mpfigabcsr{width}% in terms of \columnwidth
% {path/filename}
% {text beneath the figure.  Usually for the numbering of the subfigure}
% {place negative height, say, -4ex to reduce the gap between fig and caption}
% {left bottom right top}
\newcommand{\mpfigabcsr}[5]{
\begin{minipage}[b][1\totalheight]{#1\columnwidth}
\par\vspace{0pt}
{\centering \includegraphics[width=1\columnwidth,keepaspectratio,%
 trim=#5,clip=true]{#2}\par
\vspace*{#4}#3\par}
\end{minipage}
}



% a single figure; FIGUREHERE version
%
% usage:
% \hfig{width}
% {path/filename}
% {caption}
% {label}
\newcommand{\hfig}[4]{
\begin{figurehere}
{\centering \includegraphics[%
  width=#1,%
  keepaspectratio]{#2}
\caption{#3}
\label{#4}
\par
}
\end{figurehere}
}

% a single figure, with additional text beneath the caption.  Usually the source of the figure. FIGUREHERE version
%
% usage:
% \hfigt{width}
% {path/filename}
% {caption}
% {label}
% {additional text beneath the caption.  Usually the source of the figure.}
\newcommand{\hfigt}[5]{
\begin{figurehere}
{\centering \includegraphics[%
  width=#1,%
  keepaspectratio]{#2}%
\caption{#3}
\label{#4}
#5\par}
\end{figurehere}
}

% controlling the gap between fig and caption
% FIGUREHERE version
%
% usage:
% \hfigts{width}
% {path/filename}
% {caption}
% {label}
% {additional text beneath the caption.  Usually the source of the figure.}
% {place negative height, say, -4ex to reduce the gap between fig and caption}
\newcommand{\hfigts}[6]{
\begin{figurehere}
{\centering \includegraphics[%
  width=#1,%
  keepaspectratio]{#2}
\vspace{#6}\par
\caption{#3}
\label{#4}
#5\par}
\end{figurehere}
}

% single figure, additional text below the caption spacing control between the figure and the caption, trimming
% FIGUREHERE version
%
% usage:
% \hfigtsr{width}
% {path/filename}
% {caption}
% {label}
% {additional text beneath the caption.  Usually the source of the figure.}
% {place negative height, say, -4ex to reduce the gap between fig and caption}
% {left bottom right top} in unit of bp (= 1/72 in)
\newcommand{\hfigtsr}[7]{
\begin{figurehere}
{\centering \includegraphics[%
  width=#1,%
  keepaspectratio,%
  trim=#7,clip=true]{#2}
\vspace{#6}\par
\caption{#3}
\label{#4}
#5\par
}
\end{figurehere}
}



% main body 論文主體。建議以「章」為檔案分割的依據。
% 在下列各行,建議了
%   intro.tex, experiment.tex, theory.tex, calculation.tex, summary.tex
% 做為這幾個「章」的檔案名稱
% 實際命名方式可以隨你意
% 在撰寫各章草稿時,可以把其他章節關掉 (行首加百分號)
%
%
% this file is encoded in utf-8
% v2.03 (Feb. 26, 2013)

% 臨時定義了 fmpage: 一個加框的展示區 framed minipage
% http://brunoj.wordpress.com/2009/10/08/latex-the-framed-minipage/
\newsavebox{\fmbox}
\newenvironment{fmpage}[1]
{\begin{lrbox}{\fmbox}\begin{minipage}{#1}}
{\end{minipage}\end{lrbox}\fbox{\usebox{\fmbox}}}
%%%%%%%%%%%%%%  緒論  %%%%%%%%%%%%%%%%%%%%%
 \chapter{緒論}
\label{sec:intro}


%%%%%%%%%%%%%%%%%%%%%%%%%%%%%%%%%%%

%%%%%%%%%%%%%%  研究背景  %%%%%%%%%%%%%%%%%%%%%
\section{研究背景}
\cite{ArchLinuxARM-rpi-latest}

  % 所附的範例
%\input{intro.tex}
%\input{experiment.tex}
%\input{theory.tex}
%\input{calculation.tex}
%\input{summary.tex}

% back pages 後頁
% 包括參考文獻、附錄、自傳
% 實際內容由
%    my_bib.bib, my_appendix.tex, my_vita.tex
% 決定
% yzu_backpages.tex 此檔只提供整體架構的定義,不需更動
% 在撰寫各章草稿時,可以把此部份「關掉」,以節省無謂的編譯時間。
%
% this file is encoded in utf-8
% v2.01 (Jul. 3, 2012)

%%% 參考文獻
\newpage
\phantomsection % for hyperref to register this
\addcontentsline{toc}{chapter}{\nameRef}
\renewcommand{\bibname}{\protect\makebox[5cm][s]{\nameRef}}
%  \makebox{} is fragile; need protect
\bibliographystyle{IEEEtran}  % 使用 IEEE Trans 期刊格式
\bibliography{./content/bib/my_bib}


%%% 附錄
%%%%
% this file is encoded in utf-8
% v2.02 (Sep. 12, 2012)
%%% 每一個附錄 (附錄甲、附錄乙、...) 都要複製此段附錄編排碼做為起頭
%%% 附錄編排碼 begin >>>
\newpage 
\chapter*{附錄 A:  MATLAB / Octave 程式列表} % 修改附錄編號與你的附錄名
\phantomsection % for hyperref to register this
\addcontentsline{toc}{chapter}{附錄 A: MATLAB / Octave 程式列表} %建議此內容應與上行相同
%\setcounter{chapter}{0}  %如果用的是 TeXLive2007 則打開此行以避免錯誤 
\setcounter{equation}{0} 
\setcounter{figure}{0} 
\setcounter{footnote}{0} 
\setcounter{section}{0} 
\setcounter{subsection}{0}
\setcounter{subsubsection}{0}
\setcounter{table}{0} 
\renewcommand{\thechapter}{A} % 如果是附錄 B,則內容應為{B}
%%% <<< 附錄編排碼 end

% 附錄內容開始
%%% 納入程式源碼
\lstinputlisting[caption={matlab 程式碼列表範例},
label=lst:matlab:example,
numbers=left,
firstnumber=1,
frame=ltrb, % single lines for left, top, right, bottom; LTRB for double lines 
escapeinside={$$}, %如要在列表裡顯示特殊字元/排版效果,要把該文字串用 $$ 包夾住 (適合 C 程式碼)(原預設為 <>)
]
{appendix/example_prog_list.m}

\begin{equation}\sum_{k=1}^{n} k = \frac{n(n+1)}{2}\end{equation}

%%% 如果有附錄B、C、...,則在此繼續加上「附錄編排」碼
% 每一個附錄會自動以新頁開始


%%% 自傳
%%%\newpage
%%%\chapter*{\protect\makebox[5cm][s]{\nameVita}} % \makebox{} is fragile; need protect
%%%\phantomsection % for hyperref to register this
%%%\addcontentsline{toc}{chapter}{\nameVita}
%%%本人生於 1981 年 1 月 1 日,在桃園內壢。家裡經營電器行,上有一位姊姊。從小就喜歡拆解店裡收回的報廢家電用品,練就了一身好手藝與探究一切的好奇心。

國小就讀平鎮國小。由於把供應全校用水的抽水馬達拆開研究裝不回去,造成全校停水,廁所污穢不堪。被校長處罰掃廁所一個星期。那真是我少時年幼無知的一頁插曲。





\clearpage % to make sure all CJK characters are processed
\end{CJK*}  %%% ZZZ %%%
\end{document} 
 
